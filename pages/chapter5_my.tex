\chapter{总结与展望}

\section{总结}
当下知识图谱技术的应用日益渗透到每个人的日常生活中,从搜索引擎到推荐系统,不断探索知识图谱知识的应用与补充是未来发展不可或缺的一个研究方向。但是出于数据隐私、成本等多方面的考量,大到公司多个图谱存储服务器,小到人们每日使用的移动终端,我们无法将所有分散知识图谱添加的实体和关系完全覆盖,如何在训练集训练能够快速应用于包含未见组件的新兴知识图谱的研究是无可避免的需求。

本文采用元学习的方法在训练集中模拟了新兴知识图谱中可能出现的未见关系和未见实体,使得模型能够在训练中获得处理未见组件的能力,同时元学习的方法能够一次训练快速应用,在算力珍贵的时代能够最大提高训练的效率,降低成本消耗。对未见关系的表示能力学习上,本文通过对关系相对位置关系的应用构建出以关系为结点的图,图上的关系之间通过预定义的元关系进行连接,体现出关系的拓扑信息;知识图谱作为丰富语义信息的载体,为了对拓扑信息进行补充,在关系图的基础上,本文通过对知识图谱高层语义体现的本体信息进行嵌入作为关系语义信息的体现,然后在关系图上使用GCN对关系节点的嵌入进行拓扑信息和语义信息的学习,获得对未见关系的表达。对未见实体的表示能力学习上,本文通过对图谱的结构分析认为,相似类型的实体在邻接结构上往往表现出高度的相似性,因此采用实体的邻接关系特征聚合作为未见实体的初始化表示,避免了对未见实体周围组件的苛刻性要求。但关系的表示仍然不足以支撑对实体特征的体现,模型最后在实例知识图谱上,采用CompGCN对每一个实体和关系再次进行邻接信息的学习和更新,获得最终的各组件的嵌入。本文最后通过两个数据集上的实验和多个基准模型的比较和链接预测任务得分以及案例分析证明了本文模型的有效性。本文的工作总结如下:
\begin{enumerate}[label=\arabic*)]
  \item 在对知识图谱组件的表示学习中引入了本体信息作为知识图谱的语义信息补充,通过对拓扑信息和语义信息的结合能够有效对未见组件进行表示学习,同时采用GNN层对初步表示学习的效果进行加强,有效学习到了各组件的特征信息。
  \item 采用元学习的训练方式对模型进行训练,在元学习的单任务设定中对未见组件进行模拟,使得模型能够学习到处理未见组件的能力同时提高训练的效率降低成本。
  \item 在两个知识图谱链接预测基准数据集的基础上构建满足跨设备场景要求下的测试数据集,并与多个基准模型进行实验效果比较,通过对结果分析可知本文模型相比于其他基准模型均有不同程度的提升,证明了本文模型的有效性。
\end{enumerate}

\section{未来工作}
虽然本文在测试数据集上的效果相比于其他基准模型已经获得了明显的提升,但是在本文的整个研究过程中仍旧发现了当下模型在应用方面值得继续探究的几个研究方向:
\begin{enumerate}[label=\arabic*)]
  \item 本体嵌入的评分函数对模型效果的影响:在对本体的嵌入过程中,本文采用了目前传统KGE方法中效果最好的RotatE模型作为评分函数,但是从实验结果中我们可以发现对于DistMult和ComplEx评分函数的得分会比更简单的TransE更低,那么不同的本体嵌入方法可能对模型的效果会产生影响,是否本体嵌入评分函数和模型评分函数相对应会更获得更好的模型效果值得探究。
  \item 数据集本体信息的提取:本文模型研究本体信息对知识图谱表示学习的可用性,本体三元组是必不可少的一环,但当下的一些基准知识图谱有些并没有相应的本体三元组,如被广泛使用的FB15K-237数据集,因为其源数据集已经停止维护本体类型信息也都比较杂乱无法使用,虽然本文在处理测试数据集时采用了对实例三元组类型信息补充了本体三元组中的关系信息,但面对一个数据集如何更充分获取到其本体信息仍旧是待解决的难题。
  \item 规则信息的引入:从DB\_Ext的基准模型测试信息可以看出基于规则提取的模型在未见的实体上也能够不错的表现效果,而且规则理论上可以同时作用于实体和关系,是对未见组件表示学习进行约束的良好工具,如何将规则信息融入到本文模型更好的学习到未见组件的表示也是未来工作的一个值得期待的方向。
\end{enumerate}