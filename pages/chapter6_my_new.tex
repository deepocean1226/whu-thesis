\chapter{总结与展望}

\section{总结}
当前,知识图谱技术的应用已经深入到了人们的日常生活中,从搜索引擎到推荐系统,不断挖掘知识图谱知识的应用是人工智能发展的基础。然而,出于数据隐私和成本等多方面的考虑,大到公司多个图谱存储服务器,小到人们每日使用的移动终端,我们无法将所有分散知识图谱新添加的实体和关系完全覆盖。因此,面向跨领域知识表示学习问题的研究已成为不可避免的需求和研究方向。

本文针对跨域知识图谱的知识表示学习问题,采用元学习的方法,在训练任务中模拟跨域场景下的未见关系和未见实体,从而获得了跨域知识表示的能力。元学习方法具有训练效率高的优点,可以在算力珍贵的时代大幅降低成本消耗。对于未见关系的表示,本文依据关系间的相对位置关系构建了一个以关系为结点的图,并通过预定义的元关系将其连接起来,以学习关系的拓扑信息。此外,本文将本体信息嵌入作为图谱语义信息的补充,通过图卷积网络对关系节点进行拓扑信息和语义信息的联合学习,获得了对未见关系的表示。对于未见实体的表示,本文认为相似类型的实体具有相似的邻接结构,并采用了实体的邻接关系特征聚合作为未见实体的初始化表示。为了能够充分利用到已知实体和关系的特征信息,本文使用图神经网络对实体和关系进行邻接信息的学习和更新。通过实验、基准模型比较和案例分析,本文证明了所提出模型的有效性。总的来说,本文的工作主要包括:
\begin{enumerate}[label=\arabic*)]
  \item 提出了一个基于本体信息和元学习的跨域知识表示学习框架,采用元学习的任务设定在模型训练中对跨域场景进行模拟,并结合图的拓扑结构信息和本体语义信息对未见实体和未见关系进行建模,使得模型具备处理未见实体和关系的能力。
  \item 对于未见关系和实体,本文创新性的同时考虑关系的拓扑结构和本体语义两个方面的特征,通过在本体视图中引入拓扑元关系,并利用图网络实现拓扑关系和语义信息的联合表示。
  \item 在两个测试数据集上进行了模型实验,并将结果与多个基准模型进行了比较。实验结果表明本文模型相比于其他基准模型均有不同程度的提升,证明了本文模型的有效性。
\end{enumerate}

\section{未来工作}
虽然本文在测试数据集上的效果相比于其他基准模型已经获得了明显的提升,但是在本文的整个研究过程中仍旧发现了当下模型在应用方面值得继续探究的几个研究方向:
\begin{enumerate}[label=\arabic*)]
  % \item 本体嵌入的评分函数对模型效果的影响:在对本体的嵌入过程中,本文采用了目前传统KGE方法中效果最好的RotatE模型作为评分函数。但是从实验结果中我们可以发现对于DistMult和ComplEx评分函数的得分会比更简单的TransE更低,那么不同的本体嵌入方法可能对模型的效果会产生影响,是否本体嵌入评分函数和模型评分函数相对应会更获得更好的模型效果值得探究。
  \item 本文引入了本体信息,以增强跨域知识表示学习。其中,本体三元组是不可或缺的。然而,目前一些基准知识图谱缺少相应的本体三元组。例如,广泛使用的FB15K-237数据集,其源数据集已经停止维护,因此本体类型信息较为杂乱难以使用。如何更充分地获取数据集的本体信息仍然是待解决的难题。
  \item 在DB\_Ext数据集上的实验结果表明,基于规则提取的模型在处理未见的实体时也能表现出不错的效果。此外,规则可以同时作用于实体和关系,并对其表示学习进行约束,如何将规则信息融入到本文模型也是未来工作的一个值得期待的方向。
\end{enumerate}