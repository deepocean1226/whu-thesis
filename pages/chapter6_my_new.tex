\chapter{总结与展望}

\section{总结}
当下知识图谱技术的应用日益渗透到每个人的日常生活中,从搜索引擎到推荐系统,不断探索知识图谱知识的应用与补充是未来发展不可或缺的一个研究方向。但是出于数据隐私、成本等多方面的考量,大到公司多个图谱存储服务器,小到人们每日使用的移动终端,我们无法将所有分散知识图谱新添加的实体和关系完全覆盖。如何在源域知识图谱上训练,并且能够将表示学习能力泛化到应用于包含未见实体和关系的目标域知识图谱的研究是无可避免的需求。

面向跨域知识图谱的知识表示学习问题,本文采用了元学习的方法,在训练任务中模拟了跨域场景下的未见关系和未见实体,从而获得了模型处理未见组件的能力。元学习方法具有训练效率高的优点,可以在算力珍贵的时代大幅降低成本消耗。对于未见关系的表示能力学习,本文依据关系间的相对位置关系构建了一个以关系为结点的图,并通过预定义的元关系将其连接起来,以体现出关系的拓扑信息。此外,我们将本体信息嵌入作为关系语义信息的体现,再经过图卷积网络对关系节点进行拓扑信息和语义信息的联合学习,获得了对未见关系的表达。针对未见实体的表示能力学习,本文通过分析总结出相似类型实体在邻接结构上表现出高度相似性这一结论,并采用了实体的邻接关系特征聚合作为未见实体的初始化表示,避免了对未见实体周围节点的苛刻性要求。但聚合关系的表示仍然不足以支撑对实体特征的体现,为了能够充分利用到已知实体和关系的特征信息,我们使用CompGCN对实体和关系进行邻接信息的学习和更新,从而获得了各组件的嵌入。通过实验、基准模型比较和案例分析,本文证明了所提出模型的有效性。归纳起来,本文的工作主要包括:
\begin{enumerate}[label=\arabic*)]
  \item 为了充分捕捉到本体中实体和关系对应的语义信息,对本体三元组进行嵌入,学习到本体的向量表示。同时为了补充本体信息中的关系信息,构建了关系的位置元关系,补充了本体中的关系本体三元组。在本体三元组结构信息嵌入的基础上,基于描述文本进一步对本体嵌入加强。
  \item 在对跨域知识图谱的知识表示学习中引入了本体信息作为知识图谱的语义信息补充,通过对拓扑信息和语义信息的结合能够有效对未见实体和关系进行表示学习,同时采用复杂图卷积层对表示学习的效果进行加强,有效学习到了实体和关系的特征表示。模型总体采用元学习的任务设定对模型进行训练,在元学习的单任务中对跨域场景进行模拟,使得模型能够学习到处理未见实体和关系的能力,同时提高训练的效率,降低成本。
  \item 在两个知识图谱链接预测基准数据集的基础上构建满足跨域场景要求下的测试数据集,并与多个基准模型进行实验效果比较,通过对结果分析可知本文模型相比于其他基准模型均有不同程度的提升,证明了本文模型的有效性。
\end{enumerate}

\section{未来工作}
虽然本文在测试数据集上的效果相比于其他基准模型已经获得了明显的提升,但是在本文的整个研究过程中仍旧发现了当下模型在应用方面值得继续探究的几个研究方向:
\begin{enumerate}[label=\arabic*)]
  \item 本体嵌入的评分函数对模型效果的影响:在对本体的嵌入过程中,本文采用了目前传统KGE方法中效果最好的RotatE模型作为评分函数。但是从实验结果中我们可以发现对于DistMult和ComplEx评分函数的得分会比更简单的TransE更低,那么不同的本体嵌入方法可能对模型的效果会产生影响,是否本体嵌入评分函数和模型评分函数相对应会更获得更好的模型效果值得探究。
  \item 数据集本体信息的提取:本文模型研究本体信息对知识图谱表示学习的可用性,本体三元组是必不可少的一环。但当下的一些基准知识图谱有些并没有相应的本体三元组,如被广泛使用的FB15K-237数据集,因为其源数据集已经停止维护,本体类型信息也都比较杂乱无法使用。虽然本文在处理测试数据集时采用了对实例三元组类型信息补充了本体三元组中的关系信息,但面对一个数据集如何更充分获取到其本体信息仍旧是待解决的难题。
  \item 规则信息的引入:从DB\_Ext的基准模型测试信息可以看出基于规则提取的模型在未见的实体上也能够不错的表现效果,而且规则理论上可以同时作用于实体和关系,是对跨域知识图谱表示学习进行约束的良好设定。如何将规则信息融入到本文模型更好的学习到未见组件的表示也是未来工作的一个值得期待的方向。
\end{enumerate}