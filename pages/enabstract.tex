With the widespread application of knowledge graphs in the industrial sector, numerous knowledge-graph-based applications have entered people's vision, such as e-commerce systems, medical health management systems, etc., where the structured knowledge stored constitutes the knowledge graph of various systems in their source domains. In order to fully utilize the knowledge in the graph, knowledge representation learning embeds the entities and relationships in the graph into a low-dimensional continuous space while preserving their semantic information, which helps with various downstream tasks. With the growth of knowledge graph data and its application on personal computers, mobile devices, etc., the processing and storage of knowledge graphs have gradually been distributed across multiple different nodes. The target-domain knowledge graph in the aforementioned nodes is usually a subset of the source-domain knowledge graph, and new entities and relationships are constantly introduced during user usage. However, these introduced entities and relationships may include new and undefined entities and relationships not in the source-domain knowledge graph entity set and relationship set, and we cannot fully cover all dispersed knowledge graph newly added entities and relationships; therefore, how to train and generalize the representation learning capability onto the target-domain knowledge graph, which contains unknown entities and relationships, has become an inevitable problem in cross-domain knowledge representation.

To address this issue, this paper makes use of meta-learning algorithm ideas, simulating unseen entities and relationships on the target-domain knowledge graph through labels in the training task, to help the model acquire the capability of handling new entities and relationships in the target-domain graph, while introducing ontology information to provide semantic support for the representation learning of unseen entities and relationships. This paper mainly addresses two problems in cross-domain knowledge representation learning: (1) current representation learning models for cross-domain knowledge graphs tend to use graph structure information to learn the representation of unseen entities and relationships, failing to fully leverage the semantic information of the knowledge graph; (2) for cross-domain knowledge graphs, how to migrate source-domain meta-knowledge to the target-domain graph to achieve generalization of unseen entities and relationships.

This paper proposes a knowledge representation learning framework based on ontology information and meta-learning for cross-domain knowledge graphs, including: (1) constructing a relationship-as-node view for new relationships that appear in the target-domain knowledge graph, while modeling both the topology structure of the relationships and the semantic information of the ontology; (2) for new entities that appear on the target-domain knowledge graph, initializing their representation by aggregating their adjacency relationship features, and using graph neural networks to fully utilize the information of known entities and relationships in the knowledge graph to learn and update the vector representations of the entire target-domain entities and relationships; (3) adopting meta-learning methods to partition tasks and simulate new entities and relationships in cross-domain scenarios through labels in the training process to complete the knowledge representation learning task from the source-domain to the target-domain.

This paper verifies the proposed model through link prediction tasks and compares it with multiple benchmark models. Experimental results demonstrate the effectiveness of the proposed meta-learning ontology-enhanced cross-domain knowledge representation learning model.