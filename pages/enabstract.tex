% 英文摘要

The knowledge graph representation learning embeds the entities and relationships of the graph into a continuous low-dimensional space while maintaining its semantic information. This has been helpful for many downstream applications, such as knowledge graph completion, recommendation systems, and question-and-answer systems.

As the size of knowledge graphs continues to expand, these separate graphs are often scattered across different devices. Moreover, due to the rapid development of personal mobile terminals, each person's mobile device also has various scaled graph applications. For cost consideration and user data privacy requirements, these cross-device graphs often cannot be effectively merged and fused. In the cross-device scenario, the knowledge graph representation learning faces the following issues: how to better deal with emerging knowledge graphs in which there are unseen entities and relationships that exist across various devices with the current trained knowledge graph model.

To solve this problem, this paper uses meta-learning training methods to simulate the unseen entities and relationships for training knowledge graphs to help the model learn the ability to deal with the unseen components. In addition, ontology information is introduced to provide semantic support for the representation learning of unseen components. This mainly solves two problems in the current models: (1) The current models that deal with unseen components can only handle individual unseen entities or unseen relationships and cannot consider both as a whole. The model that represents the structural information of the unseen components lacks semantic information of the rich knowledge graph. (2) How to simulate the decentralized storage method of cross-device knowledge graphs in model training, so that the model can obtain stronger generalizability.

Based on the analysis of the above issues, this paper designs a representation learning framework based on Graph Neural Networks (GNN) and ontology information. It can both represent the features of unknown components and simulate tasks using meta-learning methods in model training to obtain generalization learning ability for emerging knowledge graph unknown components. When representing unknown relationships, ontology information reflecting the high-level semantics of the knowledge graph is embedded, and topology information and semantic information learning are combined by using GCN on the relationship graph to obtain the expression of unseen relationships. For unknown entities, the paper uses the feature aggregation of adjacent relationship characteristics for initialization and representation, and uses CompGCN to learn and update the adjacent relationship of each entity and relationship, and obtain the final embedding of each component.

The paper extracted specific test data sets based on the scene task requirements from two benchmark datasets and tested the model's link prediction task compared with many benchmark models. From the experimental results, the model had significantly improved performance. Furthermore, through multiple module ablation experiments and multi-dimensional case analysis, the paper proves the effectiveness of the model.