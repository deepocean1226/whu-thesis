With the widespread application of knowledge graphs in the industry, numerous knowledge graph-based applications have entered people's vision, such as e-commerce systems, medical and health management systems, etc. The structured knowledge stored in these applications forms the source domain knowledge graph. In order to fully utilize the knowledge in the graph, knowledge representation learning embeds the entities and relationships in the graph into low-dimensional continuous space while preserving their semantic information, which is helpful for various downstream tasks. With the growth of knowledge graph data scale and its application on personal computers, mobile devices, etc., processing and storage of knowledge graphs are gradually distributed across multiple different nodes. The target domain knowledge graph in the above-mentioned nodes is usually a subset of the source domain knowledge graph and constantly introduces other entities and relationships during the user's usage. However, these introduced entities and relationships may contain new and undefined entities and relationships that are not defined in the source domain knowledge graph entity set and relationship set, and we cannot completely cover all the newly added entities and relationships in the dispersed knowledge graph. Therefore, how to train on the source domain knowledge graph and generalize the representation learning ability to the target domain knowledge graph containing unknown entities and relationships has become an inevitable cross-domain knowledge representation problem.

To solve this problem, this paper employs the algorithmic idea of meta-learning and simulates the unseen entities and relationships on the target domain knowledge graph through labels in the training task to help the model acquire the ability to process new entities and relationships on the target domain graph. Meanwhile, ontology information is introduced to provide semantic support for the representation learning of unseen entities and relationships. This paper mainly solves two problems in cross-domain knowledge representation learning: (1) the current representation learning models for cross-domain knowledge graphs mostly use graph structure information to learn the representation of unseen entities and relationships, and do not fully utilize the semantic information of the knowledge graph; (2) for cross-domain knowledge graphs, how to migrate the meta-knowledge on the source domain knowledge graph to the target domain graph to achieve generalization of unseen entities and relationships.

This paper proposes a knowledge representation learning framework based on ontology information and meta-learning for cross-domain knowledge graphs, which includes: for new relationships appearing on the target domain knowledge graph, a view based on relationships is constructed, and both the topological structure and the semantic information of relationships are modeled as features; for new entities appearing on the target domain knowledge graph, its initial representation is obtained by aggregating the adjacency relationship features of the entity, and graph neural networks are used to fully utilize the known information of the entities and relationships in the knowledge graph to learn and update the vector representations of the entire entities and relationships in the target domain; during the model training process, the meta-learning method is used to divide tasks and simulate new entities and new relationships in the cross-domain scenario through labels, thus completing the knowledge representation learning task from the source domain to the target domain.

This paper verifies the proposed model based on the link prediction task and compares it with multiple benchmark models. The experimental results demonstrate the effectiveness of the proposed meta-learning ontology-enhanced cross-domain knowledge representation learning model.