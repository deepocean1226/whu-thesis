随着知识图谱在工业界的广泛应用,众多基于知识图谱的应用进入了人们的生活,如电商系统、医疗健康管理系统等。为了更充分利用图谱中的知识,知识表示学习将图谱中的实体和关系嵌入到低维连续空间,保留其语义信息的同时有助于各种下游任务的应用。通过知识表示学习,可以来预测用户的商品喜好或疾病诊断等。随着知识图谱数据规模的增长以及在个人电脑、手机等移动设备上的应用,知识图谱的处理和存储开始逐渐分布在多个不同的节点上。上述节点上的目标域知识图谱通常是源域知识图谱的一个子集,在用户使用过程中会不断引入其他的实体和关系。然而,这些引入的实体和关系可能包含源域知识图谱实体集合和关系集合中未定义的新的未见实体和关系,经典的知识表示方法无法获取这些未见实体和未见关系的向量表示,也就难以很好地完成知识表示任务。此外,考虑到合并的成本以及保护用户数据隐私的需求,目标域和源域知识图谱无法进行知识融合,从而形成了跨域知识表示问题。

为了解决这个问题,论文借助元学习的算法思想,在训练任务中通过标签模拟目标域知识图谱上的未见实体和关系,以帮助模型学习到处理目标域图谱新实体和关系的能力。同时引入本体信息为未见实体和关系的表示学习提供语义支持。主要解决了跨域知识表示学习中的两个问题:(1)当前处理跨域知识图谱的表示学习模型多采用图结构信息来学习未见实体和未见关系的表示,未能充分利用知识图谱其他语义信息来对表示学习进行补充。(2)面对跨域知识图谱时,如何在训练中学习到对新实体和新关系的表示能力,并且将模型的学习能力泛化到目标域知识图谱的表示学习任务上。

本文面向跨域知识图谱的知识表示学习,提出了一种基于本体信息和元学习的知识表示学习框架。主要包括:面向跨域知识表示学习中出现的新关系,采用关系图卷积网络,结合关系的拓扑信息和基于关系拓扑结构和描述文本的本体信息,进行联合学习,获得对未见关系的表示;面向跨域知识表示学习中新出现的实体,使用实体的邻接关系特征聚合获得初始化表示,再采用复杂图卷积网络,充分利用知识图谱中已知实体和关系的信息,学习和更新整个目标域的实体和关系的向量表示;在模型训练过程中,采用元学习方法划分任务并通过标签模拟跨域场景中的新实体和新关系,从而完成基于源域知识嵌入学习到目标域知识表示学习任务。

本文基于链接预测任务对跨域知识表示学习的有效性进行验证,并与多个基准模型进行比较。实验结果证明了本文提出的基于元学习本体增强的跨域知识表示学习模型的有效性。