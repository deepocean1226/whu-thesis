% 中文摘要

知识图谱表示学习将图谱的实体和关系嵌入到连续的低纬空间,同时能够保持其语义信息,有助于众多下游任务的应用,如知识图谱补全、推荐系统及问答系统等。

随着知识图谱规模的不断扩大,这些单独的图谱往往分散在不同的设备上,同时由于个人移动终端的快速发展,每个人的移动设备上也都部署有不同规模的图谱应用,出于合并成本的考量以及用户数据隐私的要求,这些跨设备的图谱往往不能进行有效的合并和知识融合。在跨设备的场景下,知识图谱表示学习面临如下的问题:在现有训练知识图谱训练出的模型如何更好处理各跨设备中存在未见实体和关系的新兴知识图谱。

为了解决这个问题,本文采用元学习训练方法来模拟训练知识图谱的未见实体和关系,以帮助模型学习处理未见组件的能力;同时引入本体信息来对未见组件的表示学习提供语义支持。主要解决了当前模型中的两个问题:(1)当前处理未见组件的模型多数只能处理单独的未见实体或者未见关系,无法兼顾。而采用结构信息对未见组件表示的模型缺乏了知识图谱丰富的语义信息。(2)跨设备下的分散知识图谱存储方式如何在模型训练中进行模拟,从而使模型获取更强大的泛化性能。

基于上述问题的分析本文设计了一个基于GNN(图神经网络)并融合本体信息的表示学习框架,既能够表示未知组件的特征,也能从模型训练中采用元学习方法对任务进行模拟,从而获得对新兴知识图谱未知组件的泛化学习能力。在对未知关系进行表示时,通过对知识图谱高层语义体现的本体信息进行嵌入,结合关系图上的GCN进行拓扑信息和语义信息的学习,获得对未见关系的表达。对于未知实体,文章采用实体的邻接关系特征聚合进行初始化的表示,并使用CompGCN对每个实体和关系再次进行邻接信息的学习和更新,获得最终各组件的嵌入。

本文在两个基准数据集上根据场景任务要求抽取出特定的测试数据集,测试了模型的链接预测任务并与多个基准模型进行对比,从实验效果上可以发现本文模型均有较好的改善。同时,本文通过多个模块的消融实验以及多个维度的案例分析证明了本文模型的有效性。

