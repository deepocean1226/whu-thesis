随着知识图谱在工业界的广泛应用,众多基于知识图谱的应用进入了人们的视野,如电商系统、医疗健康管理系统等,其中存储的结构化知识构成了各系统的源域知识图谱。为了充分利用图谱中的知识,知识表示学习将图谱中的实体和关系嵌入到低维连续空间,保留其语义信息的同时有助于各种下游任务的进行。随着知识图谱数据规模的增长以及在个人电脑、手机等移动设备上的应用,知识图谱的处理和存储开始逐渐分布在多个不同的终端上。上述终端中的目标域知识图谱通常是源域知识图谱的一个子集,在用户使用过程中会不断引入其他的实体和关系。然而,这些引入的实体和关系可能包含源域知识图谱实体集合和关系集合中未定义的、新的未见实体和关系,面临着零样本(Zero-Shot)的问题。因此,如何在源域知识图谱上进行训练,并将表示学习能力泛化到包含未知实体和关系的目标域知识图谱中,是非常重要的研究问题。
% 已经成为不可避免的跨域知识表示难题。

% 为了解决这个问题,本文借助元学习(Meta-learning)的算法思想,在训练任务中通过标签模拟目标域知识图谱上的未见实体和关系,以帮助模型获得处理目标域图谱新实体和关系的能力。同时引入本体信息,为未见实体和关系的表示学习提供语义支持。
跨域知识表示学习面临着两个关键问题:(1)当前处理跨域知识图谱的表示学习模型多采用图结构信息来学习未见实体和未见关系的表示,未能充分利用知识图谱的语义信息;(2)对于跨域知识图谱,如何将源域知识图谱上的元知识迁移到目标域图谱,以实现对未见实体和关系的泛化。

本文针对这两个问题,开展面向跨域知识图谱的知识表示学习研究,提出了基于本体信息和元学习的知识表示学习方法,主要包括:对于目标域知识图谱上出现的新关系,构建了一个以关系为节点的视图,同时建模关系拓扑结构和本体语义信息两个方面的特征;对于目标域知识图谱上新出现的实体,通过对实体的邻接关系特征聚合获得其初始化表示,并采用图神经网络,充分利用知识图谱中已知实体和关系的信息,学习和更新整个目标域的实体和关系的向量表示;在模型训练过程中,采用元学习方法划分任务并通过标签模拟跨域场景中的新实体和新关系,从而完成从源域到目标域的知识表示学习任务。

本文基于链接预测任务对提出的模型进行验证,并与多个基准模型进行比较。实验结果证明了本文提出的基于元学习本体增强的跨域知识表示学习模型的有效性。